\documentclass{article}
\usepackage[utf8]{inputenc}

\title{Security Issues in Cognitive Radio Networks}
\author{ }

\begin{document}

\maketitle


\IEEEoverridecommandlockouts


\author{\IEEEauthorblockN  \setromanfont{Soumyadeep Basu,Kumar Utkarsh, Rohit Haolader, Raushan Raj,\newline Suryasen Singh } 
\newline
\textit{IIIT Allahabad, India}}
\and
\IEEEauthorblockN{}
\IEEEauthorblockA{\textit{} \\
\textit{}}
\and
\IEEEauthorblockN{}
\IEEEauthorblockA{\textit{} \\
\textit{}}
}

\maketitle

\begin {abstract}
Cognitive radio technology is the field of
wireless communication networks that enhances the
spectrum management and ef ective utilization while
of ering many social and individual benefits. The objective
of the cognitive radio network technology is to utilize the
unutilized spectrum by primary users and fulfill the
secondary users’ demands irrespective of time and location
(any time and any place). Due to their flexibility, the
cognitive radio networks are vulnerable for numerous
threats and security problems that will af ect the
performance of the network. Little attention was given to
security aspects in cognitive radio networks that include
spectrum sensing (sensing primary user), attacks that
threaten the network at various layers and adversary ef ects
on performance due to the security threats. In this survey,
we discuss the cognitive radio networks, problems involved
in sensing and management, attacks on cognitive radio
networks, attacks on various network layers and threats on
cognitive radio networks,.

\end{abstract}



\section{Introduction}
The first idea of cognitive radio in wireless transmission
was given by Joseph Mitola in 1998. with an aim to provide
appropriate intelligence to portable devices so that they
cater to communication needs. The portable device is
required to detect open channels in the wireless spectrum
band and auto-adjust its parameters according to user
needs.
The main purpose of the sensory radio is to detect white
spaces (unused spectrum or spectrum) in the first measure
of the optimal use of that unused acquired area without
harming the main user. Sent signal detection can be done
using one or more of the techniques including a simulated
filter, power acquisition, cyclostationary element detection,
co-acquisition (sensing spectrum with the cooperative effort
for multiple sensory radiation), and an internally based
visual detection method.
Cognitive Radio Network Security includes anonymous
discovery and false user acquisition. False detection that
primary signal detection is recorded when the signal is
missing (false detection) In addition, false detection
includes malicious users pretending to be the primary user
(PU) by sending a strong signal to others users of
understanding. Misdetection involves the presence of a
primary user undetected by the cognitive user via a matched
filter.
In wireless networks, hacking and malicious attacks In
addition, security is inevitably threatened, and includes
security resources due to its nature of openness. Better
security measures ensure the soundness of the wide range.
Adoption problems apply when they operate in a hostile
environment. In case of a hostile environment, it is possibe to replicate the existing status symbols and make them
(simulate the original features) as the first user. In such
cases, integrating the official senders of the first and second
users to the spectrum sensitivity will improve the reliability
of the acquisition process.

\section{RELATED WORK}
Most of the test papers on understanding radio networks
(CRN) discuss security issues in certain aspects of the
network. Research conducted on CRN reflects the state of the
art research in certain or a few common areas. Fragkiadakis
et al [1] discussed security threats and CRN detection
strategies. This paper covers the challenges posed by
perceptual radios as well as cognitive radio networks as well
as the current state of finding parallel attacks.
Newman and Clancy [2] discussed security threats in the
classifiers classifiers. They discussed the signal separation
model, threats analysis, and feature release threats.They say
the signal separation algorithm opens up a new area of
security research related to access to a powerful spectrum
and signal segregation. They used a signal separation
algorithm to distinguish primary user (PU) and secondary
user signals. Chen et al [3] devised a defense system to
identify vulnerable users by measuring location information
and viewing signal strength. Spectrum sensitivity is also
reported by Chen et alin [4, 5]. In [4], the authors discussed
the main user simulation problem and showed disturbing
effects on comprehension radio networks.
The implications for the implementation of radio sensory in
sensory radiation were discussed by Cabric et al in 2004 [6].
The authors pointed out that the detection of a
cyclostationary element has a significant advantage between
parallel filtering and power acquisition due to its ability to
distinguish measured signals, disturbances, and low noise
levels. The process of acquiring Energy to acquire a key
signal became a major problem for security threats and
security performance focused on recent years in the analysis
of basic signal simulation. Chen et al [3 ] has used a variety
of techniques that include LocDef’s basic signal simulation
to eliminate false and inaccurate detection. Newman and
Clancy [7] discussed the security threats in signal classifiers.
They discussed the signal classifier model, threat analysis,
and threats on feature extractions. They discussed it as a new
area of research and said that it would lead to some very
interesting possibilities in this realm.


   


\section{COGNITIVE RADIO NETWORK ENVIRONMENT AND SECURITY}
One of the main requirements for understanding networks is
their ability to scan the spectral band and identify the vacant
channels available for opportunistic transmission. Since the
pri-mary user network is physically different from the
second user's network, secondary users do not receive direct
feedback from key users regarding their transfers.Usual
users must rely on individuals or their abil-ity partners to
get primary user submissions. As early users can spread
across a large area, hearing the entire spectral group
accurately is a challenging task [1] [7]. Secondary users
should rely on weak transmission signals that are weak to
measure their presence. Most research on spectrum sensor
techniques falls into three categories: transmitter detection,
co-detection and interference-based detection. The main
aim of all these methodologies is to avoid interference to
primary transmissions. The interference temperature,
amount of interference caused by all the secondary users at
a point in space, should be below a specified threshold in
the proximity of the primary users. This seems like a simple
task, it is quite hard to achieve since the location of primary
users is unknown to the secondary users. Also, when
multiple networks overlap, the users should not confuse the
transmission from other secondary users with their primary
transmissions.
The number of users along with the frequency range set
each network apart from it’s other counterparts. When a list
of spectrum bands is available to the secondary users, they
choose the most appropriate bands for themselves.
Spectrum mobility is the agility of cognitive networks to
dynamically switch between spectrum access. The
availability of vacant spectrum bands frequently changes
over time hence the designing of cognitive protocols
requires a key knowledge of spectrum mobility. Spectrum
handoff, the delay incurred during handoff, is the main
factor used in deciding the spectrum mobility. Another
important factor to be considered is the time difference
between the secondary network detecting a primary
transmission and the secondary users vacating the spectral
band and they also cause some strong interference to the
primary users.transmitter and receiver parameters. Since these are
decentralized they don’t require primary user verification,
spectrum mobility and management functions.
\section{ ATTACKS ON COGNITIVE NETWORKS}
We define cognitive network attacks as any activity that
results in (a) unacceptable conflicts with key licensed main
users or (b) missed opportunities for secondary users.
Attacks are considered strong if they involve a small
number of opponents performing small tasks but result in
high damage / loss of initial support or support to the
network.

\subsection{ A. Physical Layer based Attacks}
Physical layer is present at the bottom of the protocol
stack.In fact it is one of the simplest layers that determines
bit rate, bandwidth and channel capacity of a network.
Some of the attacks on cognitive networks that target the
physical layer:

\newline

\subsection{Sensitivity Amplifying Attack}
Attacks are considered strong if they involve a small
number to prevent interference to the main network, other
key user acquisition methods have high sensitivity to the
primary transmission (see Section 4). This leads to false
discovery and missed opportunities for secondary users. A
lucky business can increase sensitivity which is why the
number of opportunities missed by re-introducing basic
transfers to opponents doing small tasks but causing
significant damage / loss to first and second-time users on
the network.
\subsection{ B. Link Layer based Attacks}
Link layer lies above the physical layer on the network
protocol stack. This is responsible for data
compression,fragmentation and modulation. It provides the
functionality to transfer data between network entities with
a system that can possibly detect and correct errors in the
physical layer.
\newline

\subsection{Asynchronous Sensing Attack}
Instead of synchronizing the hearing function with other
secondary users in the network, the malicious second user
may transfer syncally while other secondary users perform
sensory functions. If the primary channel or other secondary
users view it as a transfer from the main user, then this may
lead to lost opportunities.
 \newline
 
 \subsection{Biased Utility Attack}
 A second malicious user can selfishly reduce user usage
limits to increase its bandwidth. If secondary users and / or
base stations are unable to contain such behavior, this could
lead to the depletion of the transmission system for other
secondary users. If a malicious user adjusts its utility
function for transmission at higher power, it will result in
other users getting less bandwidth. Some secondary users
may lose transmission ability too.

\newline

\subsection{C. Network Layer based Attacks}
The network layer in the OSI model is responsible for
routing of packets from source to destination.Every node in
the network is responsible for keeping track of traffic
(usually in the form of a table) about neighboring locations.
When a connection is to be established, every node decides
which of its neighbors should be the next link in the path to
your destination. Some of the router protocols used in
wireless for example are dynamic source routing (DSR) and
ad-hoc on demand vector (AODV) routing [32]. A
malicious node in route can disrupt a route by spreading
incorrect route information to its neighbors or by redirecting
packets incorrectly. Several channel attacks have been
found on ad-hoc wireless networks, most attacks can be
divided into two categories: retrieval of disruptive attacks
and resource use attacks which will be primarily discussed
below.
\newline

\subsection{Network Endo-Parasite Attack}
Network Endo-Parasite Attack assumes the presence of at
least one fixed or malicious node in a network. Trojanised
nodes are trying to increase interference on overloaded
chan-nels. Most of the time, the affected links are in the
pathway through the malicious nodes near the cable gate;
Attack therefore takes the name of an insect attack.Under
normal channel operation, a node provides at least uploaded
channels under the interface and transmits the latest
information to its domain neighbors. The compromised
node introduces NEPA by providing its connectors with the
most important channels. However, they do not inform their
neighbors of the change. As information is passed on to
neighbors, the network remains unaware of the change. It
causes hidden use of heavily loaded channels
\newline

\subsection{Channel Ecto-Parasite Attack}
It is a modified version of the NEPA attack in which a
captured node will launch this attack by switching interfaces
to the channel using the highest priority link. This is an easy
to perform attack but has severe effects.
\newline

\subsection{D. Transport Layer based Attacks}
It is the fourth layer in the OSI model and responsible for
end to end network communication. It also ensures that data
packets are received in the same sequence as they are sent.
UDP (User Datagram Protocol) and TCP (Transmission
Control Protocol) are two protocols that are used in this
layer. UDP is connectionless, which means we don’t have
receipt of packet delivery. TCP is connection oriented
protocol and guarantees ordered packet delivery.
\newline

\subsection{Key Depletion Attack}
Unusually high travel times and frequent relapses suggest
that hole layer transfer sessions in cognitive networks last
only a short time. This results in an increase in the number
of sessions initiated for any given application. Most
automotive security systems such as SSL and TLS establish
cryptographic keys at the beginning of each transition layer.
A higher number of times in understanding networks and
where the number of key institutions will increase the
chances of using the same key twice. Key duplication can
be used to break the basic cipher system.

\newline

\subsection{ E. Application Layer based Attacks}
Application layer stands at the top of the OSI model and
provides users basic services such as File Transfer Protocol,
email, SMB etc. Qos (Quality of Service) is one of the main
parameters in this layer and is of prime importance to
multimedia applications. Network delay in lower layers due
to unnecessary routing is the main cause of Qos degradation
in application layer attack.
\newline

\section{ Defenses against the attacks}
Some of the Cognitive Radio Network Challenges are
Spectrum sensation, Spectrum management and display
distribution.Initial signal analysis is suggested in the current
survey.For example, a malicious threat actor interprets the
primary user signal and uses the selfish use category. These
attacks can be detected by the transmitter verification
process and location verification procedures. Alternatively,
cognitive usersimulatesthe primary user for personal
gain.That is, discerning user limits These activities can be
controlled using a variety of practices and access limits.
This problem can be fixed using the honey pot database to
mislead the malicious user.
Acquisition of the main user signal is difficult when using
signal spread or changing parameters by a dangerous
user.These problems can be solved using cloud application.
The spectrum can be easily achieved through multiple users
in a collaborative way. and transactional integrity. This
method provides security and the problem of storing hidden
terminal remains. Jamming Problem, hidden storage
problem, key exchanges between bumps and malicious user
actions can be eliminated using the cloud system.Security
on cloud remains an open problem.Harmful activity may
appear outside or between users of understanding. Detection
of malpractice among cognitive users can be done using
internal detection procedures and capturing bee information.
In addition, the shortcut technology used is fitted with
communication technology that will attack the upper layer
attacks. Incorporating the cryptographic techniques or
digital signature basedprint signal identification can help
differentiate malicious users. Additional functionality is
required in this guide.The spectrum flow includes a
standard control channel, operating frequency range, and
location information. Requires the current location of the
first operating system and user functionality so that the
second user can emit the captured rays as soon as the PU
enters. Dependent visibility depends on the basic user login
and the basic user migration.The cloud application will
solve many attacks and hidden storage problems inhigh-resolution cognitive networks such as sudden first user
entry.
\newline
\section{CONCLUSION}
The literature shows that the spectrum management
schemes lack formal security models. Finally we conclude
that the threat proof mechanism is difficult and impossible.
Cognitive radio networks and wireless networks lay a huge
emphasis on threat detection and protection mechanisms.
Therefore it is recommended that threat detection
mechanisms must be developed and incorporated as a part
of spontaneous need. The study shows that security at each
layer of every network is crucial as any one of them might
lead to sabotaging the entire network, leading to a loss of
the users, both primary and secondary.
\newline



\begin{thebibliography}{00}
\bibitem{b1}A. G. Fragkiadakis, E. Z. Tragos and I. G.
Askoxylakis, “A Survey on Security Threats and
Detection Techniques in Cognitive Radio
Networks”, IEEE Communications Surveys &
Tutorials, 2013, vol. 15 , issue: 1 , pp. 428 -445.

\bibitem{b2}T. R. Newman and T. C. Clancy, “Security threats
to cognitive radio signal classifiers”, Proceedings
of the Virginia tech wireless personal
communications symposium, 2009, pp. 1-9.

\bibitem{b3}K. C. Chen, Y. J. Peng, N. Prasad, N., Liang, Y. C.
and S.Sun “Cognitive radio network architecture:
part I. general structure”, 2nd international
conference on ubiquitous information management
and communication, 2008, CRs. pp.114–119
\bibitem{b4} R. Chen and J. Park, “Ensuring Trustworthy
Spectrum Sensing in Cognitive Radio Networks”,
1st IEEE workshop on Networking Technologies
for Software Defined Radio Networks,( SDR '06),
2006, pp. 110 – 119.
\bibitem{b5}R. Chen, J. Park, Y. T. Hou and J. Reed, “ Toward
Secure Distributed Spectrum Sensing in Cognitive
Radio Networks”, IEEE Communications
Magazine, 2008,vol. 46, issue. 4, pp. 50-55.

\bibitem{b6}  D. Cabric, S. M. Mishra and R. W.
Brodersen,“Implementation Issues in Spectrum
Sensing for Cognitive Radios”, 38th Asilomar
Conference on Signals, Systems and Computers,
2004, pp. 772-776.

\bibitem{b7}T. R. Newman and T. C. Clancy, “Security threats
to cognitive radio signal classifiers”, Proceedings
of the Virginia tech wireless personal
communications symposium, 2009, pp. 1-9.

\end{thebibliography}

\end{document}
\end{document}
